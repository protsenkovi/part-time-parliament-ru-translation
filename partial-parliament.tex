В результате недавних археологических исследований на острове Paxos оказалось, что существовавший парламент функционировал вопреки приверженности законодателей к странствованиям. Копии записей парламента, хранимые каждым законодателем, были согласованы, несмотря на их частые уходы по важным делам и забывчивость посыльных. Протокол парламента жителей Paxos описывает новый подход к реализации автоматов в распределённой системе.

Представленное было обнаружено недавно за систематизацией материалов в офисе редакции TOCS. Несмотря на древность, главный редактор считал что материал не стоит публикации. Так как автор был занят работой на греческих островах, меня попросили подготовить этот материал.

Автором работы был археолог, который лишь немного интересовался компьютерными науками. К сожалению; ведь даже при условии, что цивилизация Paxon была мало изучена и непонятна, как описывает автор, их законодательная система - это исключительный пример того, как можно реализовать распределённую систему, работающую в асинхронной среде. Действительно, набор усовершенствований добавленных жителями Paxon в свой протокол оказались не известными в технической литературе.

Автор даёт краткое описание связи Парламента цивилизации Paxon и распределённых вычислений в секции 4. Информатики возможно захотят прочитать эту секцию в первую очередь. И даже раньше они могут обратиться к объяснению алгоритма для информатиков автора Lampson[1996]. Более формально алгоритм описан De Prisco [1997]. Я также добавил дополнительные комментарии о относимости древнего протокола и более современных разработок в конце четвертой секции.


\section{Проболема}
\subsection{Остров Paxos}

В начале тысячелетия, эгейский остров Paxos был процветающим торговым центром. Богатство привело к политическому совершенству, и паксоны заменили древнюю теократию парламентской формой правления. Однако торговля всё же стояла на первом месте по отношению к политической ответственности и ни один паксон горел желанием посвятить свою жизнь парламенту. Парламент должен был функционировать даже при условии, что законодатели постоянно то покидали, то приходили в законодательную палату.

Проблема управления парламентом с частичной занятостью оказывается удивительно схожей с актуальной проблемой создания отказоустойчивых распределённых систем, где законодатели соответствуют процессам, а покидание парламента --- ошибке при обработке. Решение паксонов может заинтересовать информатиков. Я представляю небольшой исторический рассказ о протоколе парламентат Paxon, продолжаемый ещё более коротким описанием его отношения к распределённым системам.

Цивилизация Paxon была уничтожена иностранным вторжением, и только недавно археологи начали раскапывать их историю. Наши знания об их парламенте в силу этих обстоятельств достаточно фрагментарны. Несмотря на то, что базовый протокол известен, мы очень плохо осведомлены во его многих деталях. Тогда как именно эти детали представляют наибольший интерес. Я возьму ответственность за предположения о том, что же в действительности совершили паксоны.

\subsection{Требования}

Главной задачей парламента заключалась в определении закона (земли), определяемого последовательностью принятых декретов. В современном парламенте бы наняли секретаря для их записи, но среди паксонов не было желающих выполнять эту роль на протяжении всего заседания. Вместо этого, каждый законодатель хранил книгу учёта в которой и делал нумерованные записи принятых декретов. Например, в книге законодателя \Lambda\iota\nu\chi\partial находилась запись:

\[
    155: Налог на оливки составляет 4 драхмы за тонну
\]

если они считала, что 155 декрет парламента был принят и устанавливал размер галога на тонну оливок в 3 драхмы. Законодатели писали несмываемыми чернилами так что записи нельзя было изменить. 

Первое требование протокола парламента заключалось в корректности записей в учётных книгах, означающей, что не в каких двух книгах не могло быть противоречивой информации. Если законодатель \Phi\iota\partial\epsilon\rho хранил в своей книге запись 

\[
    132: Лампы должны использовать только оливковое масло
\]

то никто другой не мог хранить другую запись для декрета с номером 132. Однако другие законодатели могли под этим номером записи не иметь, если он не знал был ли такой декрет принят.

Соответствие книг учёта было недостаточным, так как они могли быть попросту все пустыми. Должно было быть такое требование, которое бы гарантировало, что декреты в конце концов всё-таки были бы приняты и записаны в книги. В современном парламенте, препятствием к принятию декрета могло быть несогласие среди законодателей. Но такого не было среди паксонов, среди которых преобладала атмосфера взаимного доверия. Законодатели паксонов имели желание принять каждый декрет, выносимый на рассмотрение. Но странствующий уклад жизни был проблемой. Корректность была бы потеряна, если бы одна группа законодателей приняла декрет 

\[ 
    37: Рисовать на стенах храма запрещено 
\]

а потом ушла на банкет, в то время как другая группа вступила в зал заседания и, не зная ничего о том, что произошло, приняла противоречащий закон 
\[
    37: Свобода художественного самовыражения гарантирована 
\]

Прогресс не мог быть гарантирован, пока достаточное число законодателей не оставалось в палате достаточное количество врмени. Потому как законодатели паксонов не хотели сокращать свои дела не связанные с парламентом, быть уверенным, что когда-нибудь будет принят какой-либо декрет было невозможно. Однако, законодатели желали гарантировать, что находясь в палате, они и их помошники будут действовать без промедлений по всем вопросам пардамента. Эта гарантия позволяла вывести протокол пардамента, удовлетворяющий следующему условию прогресса:

    Если большинство законодателей\footnote{} находились в палате и никто не покидал или входил в палату в течении достаточно долгого периода времени, тогда любой декрет, вынесенный на рассмотрение будет принят и окажется записан во все книги учёта всех законодателей в палате.

\subsection{Допущения}
